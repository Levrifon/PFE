\chapter{Contexte du PFE}
\label{chap:premierchapitre}

\section{Architecture existante}
Avant notre projet, la salle TIIR n'avait pas la même configuration. En effet, le trafic réseau était envoyé vers une machine à l'étage du batiment M5, dans la salle des administrateurs. Ensuite il est filtré au niveau du Centre de Ressources Informatiques (CRI). Le CRI, gère tout le réseau à travers l'université. Ceux sont eux qui décident des politiques de filtrage sur le réseau, ils l'analysent également (en cas d'attaque par exemple). L'architecture de la salle était opaque, il n'existait pas de schéma ou de document représentant l'ancienne installation. Après discussion avec notre responsable, nous avons décider de reconfigurer les machines à zéro. Nous devions donc réinstaller une image Linux sur le poste ainsi que les logiciels nécessaire à son fonctionnement. Ceux-ci seront développés dans la partie correspondante "Technologies utilisées". L'ancienne architecture présente également divers inconvénients au niveau de l'autonomie sur le réseau dans la salle. 

\section{Technologies existantes}
Initialement, la machine des administrateurs possédait un protocole de sécurisation des communications nommé IPsec. IPsec (pour Internet Protocol Security) permet de chiffrer les communications entre pairs. Cette "pile de protocole" fut remis en cause par M. Cartigny puisque son analyse et son déboguage n'était pas simple.\\
L'installation possédait également un serveur DNS et DHCP. Ceux ci servent respectivement à faire le lien entre une adresse IP et un nom de domaine (google.com par exemple) et d'attribuer des adresses IP aux différentes machines du réseau.\\

Lorem ipsum dolor sit amet, consectetur adipiscing elit. Sed non risus. Suspendisse lectus tortor, dignissim sit amet, adipiscing nec, ultricies sed, dolor. Cras elementum ultrices diam. Maecenas ligula massa, varius a, semper congue, euismod non, mi. Proin porttitor, orci nec nonummy molestie, enim est eleifend mi, non fermentum diam nisl sit amet erat. Duis semper. Duis arcu massa, scelerisque vitae, consequat in, pretium a, enim. Pellentesque congue. Ut in risus volutpat libero pharetra tempor. Cras vestibulum bibendum augue. Praesent egestas leo in pede. Praesent blandit odio eu enim. Pellentesque sed dui ut augue blandit sodales. Vestibulum ante ipsum primis in faucibus orci luctus et ultrices posuere cubilia Curae; Aliquam nibh. Mauris ac mauris sed pede pellentesque fermentum. Maecenas adipiscing ante non diam sodales hendrerit. Ut velit mauris, egestas sed, gravida nec, ornare ut, mi. Aenean ut orci vel massa suscipit pulvinar. Nulla sollicitudin. Fusce varius, ligula non tempus aliquam, nunc turpis ullamcorper nibh, in tempus sapien eros vitae ligula. Pellentesque rhoncus nunc et augue. Integer id felis. Curabitur aliquet pellentesque diam. Integer quis metus vitae elit lobortis egestas. Lorem ipsum dolor sit amet, consectetuer adipiscing elit. Morbi vel erat non mauris convallis vehicula. Nulla et sapien. Integer tortor tellus, aliquam faucibus, convallis id, congue eu, quam. Mauris ullamcorper felis vitae erat. Proin feugiat, augue non elementum posuere, metus purus iaculis lectus, et tristique ligula justo vitae magna. Aliquam convallis sollicitudin purus. Praesent aliquam, enim at fermentum mollis, ligula massa adipiscing nisl, ac euismod nibh nisl eu lectus. Fusce vulputate sem at sapien. Vivamus leo. Aliquam euismod libero eu enim. Nulla nec felis sed leo placerat imperdiet. Aenean suscipit nulla in justo. Suspendisse cursus rutrum augue. Nulla tincidunt tincidunt mi. Curabitur iaculis, lorem vel rhoncus faucibus, felis magna fermentum augue, et ultricies lacus lorem varius purus. Curabitur eu amet (fig. \ref{fig:une-autre-image}).

%%% Local Variables:
%%% mode: latex
%%% TeX-master: "isae-report-template"
%%% End:
