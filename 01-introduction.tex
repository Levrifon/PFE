\chapter*{Introduction}
\addcontentsline{toc}{chapter}{Introduction}
\markboth{Introduction}{Introduction}
\label{chap:introduction}
%\minitoc

Le sujet "journalisation du trafic de la salle TIIR" trouve sa source dans le besoin d'autonomie des étudiants. En effet, Julien Iguchi-Cartigny, créateur de ce projet, souhaitait que l'on puisse s'affranchir du serveur proxy de l'université, afin de laisser plus de libertés pour la réalisation des travaux pratiques. Le proxy, serveur par lequel passait l'ensemble du trafic de la salle, permettait de filtrer les différentes connexions et ainsi empêcher les abus. Cependant, un tel filtrage ne peut pas être parfait, et certaines transmissions de données utiles dans le cadre des cours se trouvaient alors bloquées. Remplacer cette solution par une journalisation du trafic supprime ces contraintes pour les enseignants, mais permet de retrouver l'étudiant responsable en cas de comportement illicite (attaques sur le réseau d'entreprises, consultation de sites non-autorisés).

Ce projet de fin d'études se découpe alors en plusieurs parties. Tout d'abord, l'architecture du réseau de la salle TIIR a du être repensée. Il a fallu gérer l'installation des machines utiles pour notre projet, l'accès à internet de la salle et la répartition des adresses IP des postes de travail. Ensuite, un système a été mis en place pour enregistrer le trafic. La sauvegarde des données nécessitait un choix de technologie légère pour ne pas perturber le trafic. Les fichiers journaux sont envoyés vers une machine dédiée à leur indexation et leur analyse. Cette dernière proposera également une interface web permettant de visualiser les données plus rapidement.

%%% Local Variables:
%%% mode: latex
%%% TeX-master: "isae-report-template"
%%% End:
